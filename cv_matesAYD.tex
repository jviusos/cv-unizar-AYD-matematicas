%%%%%%%%%%%%%%%%%%%%%%%%%%%%%%%%%%%%%%%%%%%%%%%%%%%%%%%%%%%%%%%%%%%%%%
% LaTeX Modelo de CV de la Universidad de Zaragoza
% Version 0.9 (09/06/2020)
%
% Creado por:
% Juan Viu Sos
% https://jviusos.github.io/
%
% License:
% CC BY-NC-SA 3.0 (http://creativecommons.org/licenses/by-nc-sa/3.0/)
%
%%%%%%%%%%%%%%%%%%%%%%%%%%%%%%%%%%%%%%%%%%%%%%%%%%%%%%%%%%%%%%%%%%%%%%

\documentclass{cvunizar}


% ====================================================================
% Math Commands and Operators	(define new commands for your text)
% ====================================================================
\newcommand\NN{{\mathbb N}}
\newcommand\ZZ{{\mathbb Z}}
\newcommand\QQ{{\mathbb Q}}
\newcommand\RR{{\mathbb R}}
\newcommand\CC{{\mathbb C}}




% ********************************************************************
% BODY OF THE DOCUMENT
% ********************************************************************
\begin{document}

\pagestyle{styleUnizar}

% NOTAS:
% 	- Usar "\newline" para salto de linea dentro de los comandos de entrada.
%
% 	- Para crear espacio vertical, se recomiendo usar el comando \entrada{}, quien genera una linea vacía sin añadir el NºDoc.
%
% 	- Uso: \entrada[<NDoc1>][<NDoc1>]{ Texto } produce una entrada con "Texto" y documentos asociados "NDoc1-NDoc2" o "NDoc1" (si <NDoc2>=vacio). El contador de documentos continua a partir de NDoc2 (resp. NDoc1), asignando los numeros de documentos de las siguientes entradas auomaticamente si no se pasan nuevos parámetros.
%
%	- Se han establecido otros comandos de entrada por secciones adaptados a los datos que se piden.
%
%   - Los comandos \cvlibro, \cvcapitulo y \cvarticulo solo admiten un parámetro opcional para elegir NDoc.




% ====================================================================
% PORTADA
% ====================================================================

% Archivo de la firma (nombre del fichero imagen, dejar vacio si no hay)
\Firma{firma.png}

% Tipo de Plazas
\TipoPlazas{Profesor Ayudante Doctor}
\Departamento{Matem\'{a}ticas}
\Areas{
	\'{A}LGEBRA\\
	AN\'{A}LISIS MATEM\'{A}TICO\\
	DID\'{A}CTICA DE LA MATEM\'{A}TICA\\
	GEOMETR\'{I}A Y TOPOLOG\'{I}A
}


% ====================================================================
% DATOS PERSONALES
% ====================================================================

% Datos Personales
\Nombre{Henri}
\Apellidos{Poincar\'e}
\DocIdentificacion{FR29041854}
% Nacimiento
\NacFecha{29/04/1854}
\NacProvincia{Lorraine}
\NacLocalidad{Nancy}
% Residencia
\ResProvincia{Huesca}
\ResLocalidad{Jaca}
\ResCodPostal{22700}
\ResDomicilio{C/Mayor 24}
% Contacto
\Tlfno{XXXXXXXXX}
\Email{hpoincare87@facdessciences.org}



% ====================================================================
% DATOS PERSONALES
% ====================================================================


\Centro{ETSI de Caminos, Canales y Puentes. Universidad Polit\'ecnica de Madrid.} % Empresa/Centro/Entidad

\Actividad{Profesor docente e investigador.} % Actividad principal

\Categoria{Profesor Ayudante Doctor.} % Categoria Profesional

% Comentar la siguiente linea si no se quiere este bloque
% \bloqueSituacionActual




% ====================================================================
% BLOQUES
% ====================================================================


\begin{bloque}{FORMACI\'{O}N ACAD\'{E}MICA}
	%
%
% ====================================================================
\seccion{%
	Titulaciones de licenciatura o grado {\sf (acompañar certificaciones académicas)}%
}{%
	\underline{Tipo}\hfill\hfill\underline{Centro/Universidad de expedici\'{o}n}\hfill\underline{Fecha de expedici\'{o}n}\hfill
}%
	
	
	\cventrada[1][2]{Licenciature en Sciences/Ing\'enieur de Mines}{École polytechnique de Paris}{02/09/1876}{}
	
	\entrada{}
	
	
	\cajaSI{Premio extraordinario fin de carrera}




% ====================================================================
\seccion{%
	M\'{a}ster y doctorado
}{}
	
\subseccion{%
	{\bf M\'{a}steres y estudios de postgrado} (acompañar certificaciones académicas)
}{%
	\underline{Tipo}\hfill\hfill\underline{Centro/Universidad de expedici\'{o}n}\hfill\underline{Fecha de expedici\'{o}n}\hfill
}%
			
	\cventrada{Máster en Iniciación a la Investigación en Matemáticas}{Universidad de Zaragoza}{21/09/1877}{}
	
	\entrada{}
	\entrada{}





\subseccion{%
	{\bf Cursos de doctorado/L\'{i}neas de investigaci\'{o}n} (Plan de Doctorado 1998) (acompañar certificaciones académicas en las que conste el t\'{i}tulo del programa de doctorado, el plan de doctorado correspondiente, Universidad, créditos y fecha de obtenci\'{o}n)
}{ }

	\entrada{}
	\entrada{}
	
	\cajaNO{Diploma de estudios avanzados o suficiencia investigadora}





\subseccion{%
	{\bf Tesis doctoral} (acompañar certificaci\'{o}n académica en las que conste el t\'{i}tulo de la tesis, del programa de doctorado, la calificaci\'{o}n obtenida y la fecha de expedici\'{o}n del t\'{i}tulo)
}{ }
		
	\entrada[]{%
	Título de la tesis doctoral, programa, calificación y fecha de expedición: 
	}%

	\cventrada[5][6]{``Sur les propriétés des fonctions définies par les équations aux différences partielles''}{\'Ecole polytechnique de Paris}{Programme de Doctorat en Sciences}{“Très honorable”, 02/08/1879}
	\cvdescription[]{Director de doctorado: Prof. Charles Hermite.}
	
	\cajaSI{Premio extraordinario de doctorado}
	



% ====================================================================
\seccion{%
	Cursos, seminarios y talleres en los que haya participado como asistente.\newline {\sf\normalsize S\'{o}lo se valorar\'{a}n los relacionados con el \'{a}rea de conocimiento y relevantes para el perfil de la plaza }
}{ }
	
%	\entrada{}
	
	\cventrada{School ``XX School of Mathematics Lluís Santaló 1880: p-Adic Analysis, Arithmetic and Singularities''}{Universidad Internacional Menendez Pelayo, Santander}{1880}{}

	
	\entrada{}
	
% ====================================================================
\seccion{%
	Otros méritos de formaci\'{o}n {\sf (acompañar certificaciones académicas)}
}{ }
	
	\entrada{{\bf Acreditación} ``Maître de conférences en Section 25 - Mathématiques'' para acceder a los cuerpos docentes en Francia.}
	
% 	\entrada{}

\end{bloque}
\newpage


\begin{bloque}{DOCENCIA}
	%
%
% ====================================================================
\seccion{%
	Docencia seg\'{u}n el Plan de Ordenaci\'{o}n Docente de la Universidad de Zaragoza, o en su caso, el equivalente de otra universidad p\'{u}blica o privada.
}{%
	\setlength{\tabcolsep}{12pt}
	\begin{tabular}{ccccccc}
	\underline{Categoría} &
	\makecell{\underline{Régimen}\\ \underline{dedicación}} &
	\makecell{\underline{Universidad Pública o}\\ \underline{Privada}} &
	~~~~\underline{Área de conocimiento}~~~~ &
	\makecell{\underline{Fecha}\\ \underline{inicio}} &
	\underline{Fecha fin}
	\end{tabular}
	
	\vspace*{1em}
%	\vspace*{-1em}
	\color{blue!80!black}{}Utilice dos filas para cada uno de los cursos académicos en los que haya aportado experiencia docente universitaria:
	una fila para nombrar la experiencia docente e indicar el documento acreditativo de la docencia, y otra fila debajo de 	la anterior para indicar la calificaci\'{o}n obtenida en la evaluaci\'{o}n de la docencia e indicar el documento acreditativo.
}

 	\cvdescription[]{\small *AYD = Prof. Ayudante Doctor, PI = Prof. Interino (o equivalente), PA = Prof. Asociado (o equivalente)}%
	

	\cvdocencia{Équations différentielles I}{AYD, Teor\'ia y Ejercicios, 6ECTS}{\'Ecole polytechnique de Paris}{License en G\'enie de Mines}{S3, curso 1876}{1877}
	\cvdescription[11][12]{Las encuestas correspondientes a ambos grados muestran una valoraci\'on media de la asignatura de 8.76/10 y 8.01/10, as\'i como 9.35/10 y 8.88/10 sobre mi labor docente, siendo mi tendencia a superar ligeramente el horario de clase lo que m\'as me penaliza en la nota.}
	

	
	\entrada{}
	\cvdescription[]{}

	

	
% ====================================================================
\seccion{%
	Participación en proyectos de innovación docente
}{ }%
	
	\entrada{}
	\entrada{}



% ====================================================================
\seccion{%
	Formación para la docencia universitaria\newline
	{\normalsize\sf Sólo se valorarán los relacionados con el área de conocimiento y relevantes para el perfil de la plaza}
}{ }%
	
	\entrada{}
	\entrada{}

% ====================================================================
\seccion{%
	Cursos, talleres y seminarios impartidos por el concursante
}{ }%
	
	
	\entrada{}
	\entrada{}

\end{bloque}
\newpage
	

\begin{bloque}{PUBLICACIONES}
	

% ====================================================================

\rowband{c}{%
	Claves: {\bf L} (Libro completo). {\bf CL} (Capítulo libro). {\bf A} (Artículo).
}%
\rowbandTabular{l}{p{0.27\linewidth}|p{0.27\linewidth}|p{0.42\linewidth}}{%
	 {\bf Mérito: L} (Libro completo).\newline
	 Autor/es (por orden):\newline
	 Título:\newline
	 Editorial:\newline
	 Año de publicación:
	 & 
	 {\bf Mérito: CL} (Capítulo libro).\newline
	 Autor/es (por orden):\newline
	 Título del capítulo:\newline
	 Autores del libro:\newline
	 Editorial:\newline
	 Año de publicación:
	 & 
	 {\bf Mérito: A} (Artículo).\newline
	 Autor/es (por orden):\newline
	 Título:\newline
	 Revista, Volumen, Número, Páginas\newline
	 Año de publicación:\newline
	 Indicadores de calidad:\newline
	 Categoría(s) del JCR en que está clasificada la revista\newline
	 Índice de impacto JCR y cuartil(es)\newline
	 Categoría(s) del SJR (SCOPUS) en que está clasificada la revista\newline
	 Índice de impacto SJR y cuartil(es)\newline
	 Otros indicadores de calidad:
	 %
}%



% ====================================================================
\seccion{%
	Libros y capítulos de libro
}{ }%

	\cvlibro{%
		Henri Poincar\'e
	}{%
		Science and method
	}{%
		Dover Publications, Inc., Mineola, NY
	}{%
		2003
	}%

	\cvcapitulo{%
		Henri Poincar\'e
	}{%
		Periodic and asymptotic solutions
	}{%
		Henri Poincar\'e; Vladimir Arnol'd
	}{%
		American Institute of Physics, New York
	}{%
		1993
	}%




% ====================================================================
\seccion{%
	Publicaciones en revistas científicas
}{ }%

    \cvarticulo{%
		Henri Poincar\'e
	}{%
		Sur les {F}onctions {U}niformes qui se reproduisent par des
              {S}ubstitutions {L}in\'{e}aires
	}{%
		Mathematische Annalen, 20, 52-53
	}{%
		1882
	}{%
		Mathematics (miscellaneous)
	}{%
		1.435 (Q1)
	}{%
		Mathematics (miscellaneous)
	}{%
		2.51 (Q1)
	}%


	
	\entrada{}


% ====================================================================
\seccion{%
	Otras publicaciones
}{ }%

	

	\entrada{}

\end{bloque}
\newpage


\begin{bloque}{PARTICIPACI\'{O}N EN PROYECTOS Y CONTRATOS DE INVESTIGACI\'{O}N}
	% ====================================================================
\seccion{%
	Participación en proyectos de investigación obtenidos en convocatorias públicas y competitivas, en especial los financiados mediante programas regionales, nacionales o europeos.
}{%
	\setlength{\tabcolsep}{15pt}
	\hspace*{-2em}
	\begin{tabular}{ccccccc}
	\makecell{\underline{Título del}\\ \underline{proyecto}}~~~~~~~~~~~ &
	\underline{Organismo} &
	\underline{Fecha inicio} &
	\underline{Fecha fin} &
	\makecell{\underline{Investigador}\\ \underline{principal}} &
	\makecell{\underline{Tipo de}\\ \underline{participación}}
	\end{tabular}
}%

			
	\cvproyecto{Géométrie, combinatoire et topologie}%
		{Ambassade de France au Japon, Ministère des affaires étrangères}%
		{01/01/1878}{31/12/1880}%
		{Charles Hermite}%
		{Investigador Colaborador}%			

	\entrada{}


% ====================================================================
\seccion{%
	Participación en contratos de investigación de especial relevancia en empresas o con la administración pública.
}{%
	\setlength{\tabcolsep}{15pt}
	\hspace*{-2em}
	\begin{tabular}{ccccccc}
	\makecell{\underline{Título del}\\ \underline{proyecto}}~~~~~~~~~~~ &
	\underline{Organismo} &
	\underline{Fecha inicio} &
	\underline{Fecha fin} &
	\makecell{\underline{Investigador}\\ \underline{principal}} &
	\makecell{\underline{Tipo de}\\ \underline{participación}}
	\end{tabular}
}%
	
	\entrada{}


\end{bloque}
\newpage


\begin{bloque}{RESTANTE ACTIVIDAD INVESTIGADORA}
	% ====================================================================
\seccion{%
	Participación en congresos y conferencias científicas nacionales e internacionales, presentando ponencias o comunicaciones
}{%
	\underline{Autores}\hfill\hfill
	\underline{Título}\hfill\hfill
	\underline{Tipo de participación}\hfill
	\underline{Congreso}\hfill
	\underline{Lugar celebración/año}\hfill
}%
	
	\cvcharla{Titulo}{Seminario}{Universidad}{Pais, A\~no}
	
	\cvminicurso{Titulo}{Seminario}{Universidad}{Pais, A\~no}
	
	\cvshortcomm{Titulo}{Seminario}{Universidad}{Pais, A\~no}
	
	\cvposter{Titulo}{Seminario}{Universidad}{Pais, A\~no}
	
	\entrada{}





% ====================================================================
\seccion{%
	Estancias en centros nacionales o extranjeros de investigación
}{%
	\underline{Centro}\hfill\hfill
	\underline{Localidad}~~
	\underline{País}\hfill
	\underline{Año}~~~~
	\underline{Duración}~~~~
	\underline{En calidad de}\hfill\hfill
	\underline{Tema}\hfill\hfill
}%

	\cvestancia{Universidad}{Lugar}{Fechas, duraci\'on}{en calidad de}{Tema}
	
	\entrada{}

  



% ====================================================================
\seccion{%
	Becas de investigación disfrutadas, así como otras becas y ayudas de carácter competitivo relacionadas con la participación en programas regionales, nacionales o europeos de investigación
}{%
	\underline{Denominación}\hfill\hfill
	\underline{Organismo que la concede}\hfill
	\underline{Fecha inicio}\hfill
	\underline{Fecha fin}\hfill\hfill
}%

	\cventrada{Nombre proyecto o beca}{Organismo}{Finicio -- Ffinal}{}
	
	\entrada{}


\end{bloque}
\newpage
    
    

\begin{bloque}{%
	OTROS M\'{E}RITOS {\Large no valorados en los apartados anteriores}\newline
	{\large Si se incluyen m\'{a}s de 10 méritos, s\'{o}lo se considerar\'{a}n los diez primeros}%
	}%
	


\preambNDoc{}
    \entradaEnum{Acreditación ``Profesor Ayudante Doctor'' por parte de la Agencia Nacional de Evaluaci\'on de la Calidad y Acreditaci\'on, de fecha 12 de diciembre de 1885.}
	
	\entradaEnum{Acreditación ``Maître de conférences en Section 25 - Mathématiques'' para acceder a puestos de profesorado universitario permanente en Francia.}
	
	\entradaEnum{Fluente en Inglés Nivel A2 (FCE A2, 1885)}

	\entradaEnum{Fluente en Alem\'an Nivel B2}
	
	\entradaEnum{Fluente en Francés (lengua materna)}
	
% 	\entradaEnum{}
	

\end{bloque}





\end{document}
