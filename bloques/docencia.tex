%
%
% ====================================================================
\seccion{%
	Docencia seg\'{u}n el Plan de Ordenaci\'{o}n Docente de la Universidad de Zaragoza, o en su caso, el equivalente de otra universidad p\'{u}blica o privada.
}{%
	\setlength{\tabcolsep}{12pt}
	\begin{tabular}{ccccccc}
	\underline{Categoría} &
	\makecell{\underline{Régimen}\\ \underline{dedicación}} &
	\makecell{\underline{Universidad Pública o}\\ \underline{Privada}} &
	~~~~\underline{Área de conocimiento}~~~~ &
	\makecell{\underline{Fecha}\\ \underline{inicio}} &
	\underline{Fecha fin}
	\end{tabular}
	
	\vspace*{1em}
%	\vspace*{-1em}
	\color{blue!80!black}{}Utilice dos filas para cada uno de los cursos académicos en los que haya aportado experiencia docente universitaria:
	una fila para nombrar la experiencia docente e indicar el documento acreditativo de la docencia, y otra fila debajo de 	la anterior para indicar la calificaci\'{o}n obtenida en la evaluaci\'{o}n de la docencia e indicar el documento acreditativo.
}

 	\cvdescription[]{\small *AYD = Prof. Ayudante Doctor, PI = Prof. Interino (o equivalente), PA = Prof. Asociado (o equivalente)}%
	

	\cvdocencia{Équations différentielles I}{AYD, Teor\'ia y Ejercicios, 6ECTS}{\'Ecole polytechnique de Paris}{License en G\'enie de Mines}{S3, curso 1876}{1877}
	\cvdescription[11][12]{Las encuestas correspondientes a ambos grados muestran una valoraci\'on media de la asignatura de 8.76/10 y 8.01/10.}
	

	
	\entrada{}
	\cvdescription[]{}

	

	
% ====================================================================
\seccion{%
	Participación en proyectos de innovación docente
}{ }%
	
	\entrada{}
	\entrada{}



% ====================================================================
\seccion{%
	Formación para la docencia universitaria\newline
	{\normalsize\sf Sólo se valorarán los relacionados con el área de conocimiento y relevantes para el perfil de la plaza}
}{ }%
	
	\entrada{}
	\entrada{}

% ====================================================================
\seccion{%
	Cursos, talleres y seminarios impartidos por el concursante
}{ }%
	
	
	\entrada{}
	\entrada{}
